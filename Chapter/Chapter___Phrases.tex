\chapter{Phrases 短语}
% \section{研究背景}
% \subsection{\LaTeX}
% \subsubsection{\LaTeX}

% 本模板为北京邮电大学研究生学位论文模板\cite{ddpg}。

% \subsection{关于参考文献及引用格式}

% 目前,北邮官方给出的《北京邮电大学关于研究生学位论文格式的统一要求》中,对于参考文献的书写及引用格式,同我国国家标准《GB/T 7714-2015》\cite{gbt7714}不完全一致。本模板采用Zeping Lee大神 \href{https://github.com/zepinglee}{(https://github.com/zepinglee)}所实现的gbt7714样式,基本保持同国标一致(因而同我校官方要求会有差别)。建议直接按照本模板所提供样式使用即可(保证全文一致与美观即可)。

% \section{本章小结}


\begin{table}[!ht]
	\centering
	% \caption{数学符号含义表(标题及序号置于表的正上方)}
	\label{chap:notation}
	\begin{tabular}{|c|c|}
		\hline
		% \rowcolor{LightSteelBlue}
        \textbf{短语} & \textbf{含义} \\
        \hline

        \rowcolor{LightGrey}
		\multicolumn{2}{|c|}{交通} \\
        \hline
        maintain the non-collision insurances & 保持非碰撞保险\\
        \hline
        handle heavy traffic & 处理拥挤的交通\\
        \hline
        under congested scenario & 在拥挤的场景\\
        \hline

        \rowcolor{LightGreen}
		\multicolumn{2}{|c|}{通信} \\
        \hline
        high communication overheads & 高通信开销 \\
        \hline

        \rowcolor{LightSteelBlue}
		\multicolumn{2}{|c|}{计算} \\
		\hline
		relieve the computational load & 释放计算负载\\
        \hline
        real-time fashion & 实时的方式\\
        \hline
        %  & \\
        % \hline
        %  & \\
        % \hline
        
	\end{tabular}
\end{table}